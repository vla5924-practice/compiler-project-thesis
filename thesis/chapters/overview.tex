\newpage
\section{Обзор предметной области}

Компилятор --- программа, способная получать на вход код программы, написанной на одном (исходном) языке и транслировать его в код эквивалентной программы на другом (целевом) языке.
Важной ролью компилятора является обнаружение ошибок во время трансляции кода исходной программы и сообщение о них пользователю.
Если целевая программа является машиночитаемой (исполняемой компьютером), она может быть вызвана пользователем для обработки входных и получения выходных данных.
Этот подход используется в таких языках программирования, как Си, C++, Go и т. д.

Интерпретатор -- это другой распространённый вид программ для обработки языков.
Вместо создания целевой программы в процессе трансляции интерпретатор непосредственно выполняет операции, указанные в коде исходной программы, с использованием предоставленных пользователем входных данных.
Интерпретируемыми языками являются, например, Python, PHP, JavaScript и т. д.

Стоит отметить, что целевая программа, предварительно транслированная с помощью компилятора, выполняется, как правило, быстрее в сравнении с исполнением при помощи интерпретатора.
Интерпретатор, тем не менее, обычно может предоставить более подробную диагностику в случае возникновения ошибок, чем это способен сделать компилятор, поскольку интерпретация подразумевает выполнение программы «построчно».

Вдобавок к описанному способу компиляции, называемому AOT-компиляцией (от. англ. ahead-of-time - заранее), существует подход, называемый JIT-компиляцией (от англ. just-in-time - во время).
В этом случае компиляция происходит непосредственно перед началом исполнения программы.
Технология JIT основывается на компиляции байт-кода и динамической компиляции.

Для разработки компиляторов существуют различные инструменты, облегчающие создание компонентов компилятора, например:

\begin{itemize}
    \item генераторы синтаксических анализаторов - для создания синтаксических анализаторов на основе имеющегося описания грамматики языка в некоторой форме (например, GNU Bison, yacc);
    \item генераторы лексических анализаторов - для создания лексических анализаторов на основе имеющегося описания лексем (слов) языка в виде набор регулярных выражений (например, flex);
    \item генераторы генераторов кода - для создания генераторов кода на основе имеющегося набора правил трансляции каждой операции на некотором промежуточном языке в машинный код, распознаваемый целевой платформой (исполнителем) (например, LLVM);
    \item инструменты для анализа графов потоков управления - для упрощения сбора информации о продвижении данных в множестве путей выполнения программы, может использоваться для оптимизации программы во время компиляции (например, aiSee);
    \item наборы инструментов для создания компиляторов, предоставляющие вспомогательные средства для реализации различных компонентов (например, LLVM, Coco/R).
\end{itemize}
