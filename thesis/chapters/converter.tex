\newpage
\section{Построение дерева операций}
\label{sec:converter}

Для перевода синтаксического дерева в дерево операций был разработан модуль, названный \textit{конвертером}.
В его задачи, помимо, собственно, построения дерева операций, входит выявление ошибок, которые могут быть обнаружены на текущем этапе.

\subsection{Концепция используемого подхода}
\label{sec:converter_concept}

Общая концепция реализованного построения дерева операций состоит в рекурсивном вызове особых вспомогательных функций в зависимости от типа текущего узла в анализируемом синтаксическом дереве.
Каждая такая функция отвечает за разбор конкретной языковой конструкции (или нескольких конструкций), представленной узлом входного дерева, может как просматривать дочерние или соседние узлы, так и редактировать строящееся дерево операций путем добавления новых узлов с помощью вспомогательной инфраструктуры.

Поскольку структура каждого узла синаксического дерева явно определена его видом, функции рекурсивно обращаются к узлам дерева и на основании их содержимого добавляют операции.
В процессе такого обхода могут возникать исключительные ситуации, когда значения, хранящиеся в некоторых узлах дерева, являются некорректными.
Такие значения могу возникнуть, например, вследствие обращения к несуществующим переменным или недопустимого порядка операций в тексте компилируемой программы.
В этих случаях функци могут возбуждать ошибки, многие из которых, тем не менее, являются восстановимыми.
Для этого может быть достаточно, например, прервать анализ текущего узла синаксического дерева и перейти к следующему.

\subsection{Список вспомогательных функций}
\label{sec:converter_functions}

Функции для перевода узла синаксического дерева конкретного вида можно разделить на два вида:

\begin{itemize}
    \item \(V\)-функции, названия которых начинаются со слова \verb|visit|.
          Особенность \(V\)-функций состоит в том, что они в результате просмотра узла возвращают "вычисленное" значение и используются, как правило, для обхода арифметико-логических выражений.
    \item \(P\)-функции, названия которых начинаются со слова \verb|process|.
          Такие функции обходят узлы, которые по своей сути не являются частями выражений (например, конструкции с циклами и заголовки функций).
\end{itemize}

Сведения о реализованных \(P\)-функциях приведены в таблице~\ref{tab:converter_process_node}, а \(V\)-функции представлены в таблице~\ref{tab:converter_visit_node}.

\begin{table}[h]
    \centering
    \caption{\(P\)-функции конвертера}
    \label{tab:converter_process_node}
    \begin{tabular}{lp{4cm}p{6cm}}
        \toprule
        \textbf{Название}                 & \textbf{NodeType\footnotemark[2]} & \textbf{Назначение}          \\
        \midrule
        \verb|processProgramRoot|         & \verb|ProgramRoot|                & Разбор от начала программы   \\
        \verb|processFunctionDefinition|  & \verb|FunctionDefinition|         & Разбор объявления функции    \\
        \verb|processBranchRoot|          & \verb|BranchRoot|                 & Разбор очередного блока кода \\
        \verb|processVariableDeclaration| & \verb|VariableDeclaration|        & Разбор объявления переменной \\
        \verb|processReturnStatement|     & \verb|ReturnStatement|            & Разбор возврата значения     \\
        \verb|processWhileStatement|      & \verb|WhileStatement|             & Разбор цикла с предусловием  \\
        \verb|processIfStatement|         & \verb|IfStatement|                & Разбор ветвления             \\
        \verb|processForStatement|        & \verb|ForStatement|               & Разбор цикла со счетчиком    \\
        \bottomrule
    \end{tabular}
\end{table}

\begin{table}[h]
    \centering
    \caption{\(V\)-функции конвертера}
    \label{tab:converter_visit_node}
    \begin{tabular}{lp{4cm}p{6cm}}
        \toprule
        \textbf{Название}               & \textbf{NodeType\footnotemark[2]} & \textbf{Назначение}                        \\
        \midrule
        \verb|visitExpression|          & \verb|Expression|                 & Разбор арифметико-логических выражений     \\
        \verb|visitIntegerLiteralValue| & \verb|IntegerLiteralValue|        & Разбор целочисленного литерала             \\
        \verb|visitBooleanLiteralValue| & \verb|BooleanLiteralValue|        & Разбор булевого литерала                   \\
        \verb|visitFPLiteralValue|      & \verb|FPLiteralValue|             & Разбор числового литерала с дробной частью \\
        \verb|visitStringLiteralValue|  & \verb|StringLiteralValue|         & Разбор строкового литерала                 \\
        \verb|visitBinaryOperation|     & \verb|BinaryOperation|            & Разбор бинарной операции                   \\
        \verb|visitVariableName|        & \verb|VariableName|               & Разбор образения к переменной              \\
        \verb|visitFunctionCall|        & \verb|FunctionCall|               & Разбор вызова функции                      \\
        \verb|visitListAccessor|        & \verb|ListAccessor|               & Разбор доступа к элементу списка           \\
        \verb|visitUnaryOperation|      & \verb|UnaryOperation|             & Разбор унарной операции                    \\
        \bottomrule
    \end{tabular}
\end{table}

Для удобства работы с деревом операций при реализации перечисленных функций каждая из них принимает объект \textit{контекста} в качестве аргумента.
Этот объект представлен структурой \verb|ConverterContext|, и содержательная часть ее объявления выглядит следующим образом.

\begin{lstlisting}[language=C++, caption=Объявление структуры ConverterContext]
struct ConverterContext {
    struct LocalVariable {
        Value::Ptr value;
        // ...
    };
    Operation::Ptr op;
    unordered_map<string, Type::Ptr> functions;
    forward_list<unordered_map<string, LocalVariable>> variables;
    Builder builder;
    // ...
};
\end{lstlisting}

\footnotetext[2]{Под NodeType подразумевается значение поля узла синтаксического дерева, характеризующего его вид.}

Таким образом, структура должна хранить в себе:

\begin{itemize}
    \item операцию, внутрь которой должна выполняться дальнейшая вставка операций;
    \item список типов функций;
    \item список переменных с учетом вложенности блоков кода;
    \item объект класса, управляющего вставкой операций.
\end{itemize}

\subsection{Основной класс Converter}

В программной реализации для работы с конвертером представлен класс Converter, который содержит единственный статический метод \verb|process|.
Этот метод получает на вход синтаксическое дерево (объект типа \verb|SyntaxTree|), а возвращает построенное дерево операций (объект типа \verb|Program|).

\begin{lstlisting}[language=C++, caption=Объявление класса Converter]
class Converter {
  public:
    Converter() = delete;
    Converter(const Converter &) = delete;
    Converter(Converter &&) = delete;
    ~Converter() = delete;

    static Program process(const SyntaxTree &tree);
};
\end{lstlisting}

Благодаря используемому подходу (см. п.~\ref{sec:converter_concept}) реализация метода \verb|process| довольно проста:

\begin{enumerate}
    \item Создается новый (пустой) объект типа \verb|Program|, представляющий выходное дерево, и в качестве корневого в него добавляется узел \verb|ModuleOp| (см. п.~\ref{sec:optree_operations}).
    \item Создается контекст (см. п.~\ref{sec:converter_functions}), в него помещается корневой узел, и он же выбирается в качестве узла, в который будут вставляться операции.
    \item Вызывается \(P\)-функция \verb|processProgramRoot| для запуска разбора от начала программы, в которую передается созданный контекст.
          В дальнейшем эта функция рекурсивно вызывает внутри себя другие функции, которые, в свою очередь, также выполняют рекурсивные вызовы.
          Таким образом, осуществляется обход всего синтаксического дерева и построение полного дерева операций.
    \item После построения дерево операций возвращается в качестве выходного значения.
\end{enumerate}
