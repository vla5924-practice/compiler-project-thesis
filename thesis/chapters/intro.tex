\newpage
\section*{Введение}
\addcontentsline{toc}{section}{Введение}

В 1957 году появился первый в мире компилятор языка Фортран, который был создан Д. Бэкусом. Компилятор позволял генерировать достаточно быстрый объектный код. До этого времени процесс создания компиляторов не был формализован, и отсутствовало понимание, что является основой компиляторов.

С развитием формальной теории языков и математической лингвистики стали появляться новые все более сложные языки программирования. Также развивались понятия генерации объектного кода, машинно-зависимых инструкций. Хотя компьютеры и их составляющие из года в год становятся лучше и лучше, основы построения компиляторов остаются неизменными.

Итак, настоящая работа будет посвящена разработке компилятора.

\newpage
\section*{Постановка задачи}
\addcontentsline{toc}{section}{Постановка задачи}

Требуется разработать и реализовать упрощенный компилятор Python-подобного статически типизированного языка с поддержкой основных языковых конструкций.  Для достижения цели проекта были поставлены следующие задачи:

\begin{itemize}
    \item определить подмножество реализуемого синтаксиса языка Python;
    \item изучить литературу о структуре и способах разработки существующих компиляторов, а также моделях исполнения программ;
    \item разработать архитектуру модулей компилятора и организацию взаимодействия между ними для выбранного языка;
    \item реализовать описанные модули, произвести их тестирование.
\end{itemize}

Разработка проекта велась коллективом в составе Власова М. С., Тактаева А. А., Шагова М. А., студентов группы 381806-1. Разделы 1, 2, 3, 6 настоящей работы отражают результаты, полученные совместно с равным вкладом всех участников коллектива. Разделы 4, 5 отражают личный вклад автора в проект.

В рамках работы были подробно описаны и реализованы следующие модули:

\begin{itemize}
    \item синтаксический анализатор;
    \item генератор промежуточного представления LLVM IR;
\end{itemize}

а также сопутствующие структуры данных.
