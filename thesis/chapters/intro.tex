\newpage
\section*{Введение}
\addcontentsline{toc}{section}{Введение}

В 1957 году появился первый в мире компилятор языка Фортран, который был создан группой программистов под руководством Джона Бэкуса в корпорации IBM.
Компилятор позволял генерировать достаточно быстрый объектный код.
До этого времени процесс создания компиляторов не был формализован, и отсутствовало понимание, что является основой компиляторов.

С развитием формальной теории языков и математической лингвистики стали появляться новые все более сложные языки программирования.
Также развивались понятия генерации объектного кода, машинно-зависимых инструкций.
Хотя компьютеры и их составляющие из года в год становятся лучше и лучше, основы построения компиляторов остаются неизменными.

Итак, настоящая работа будет посвящена разработке компилятора.

\newpage
\section*{Постановка задачи}
\addcontentsline{toc}{section}{Постановка задачи}

Требуется модернизировать компилятор Python-подобного статически типизированного языка, разработанный в рамках серии работ <<Разработка компилятора>>\footnotemark, внедрив поддержку новых языковых конструкцих и усовершенствовав промежуточное представление.
\footnotetext{
    Серия включает в себя следующие работы:
    <<Разработка компилятора. Синтаксический анализатор как компонент компилятора. Генератор промежуточного представления программы LLVM IR>> (выпускная квалификационная работа бакалавра, Власов М.~С., 2022),
    <<Разработка компилятора. Препроцессор. Абстрактное синтаксическое дерево. Оптимизирующий анализатор>> (выпускная квалификационная работа бакалавра, Тактаев А.~А., 2022),
    <<Разработка компилятора. Лексический, семантический и оптимизирующие анализаторы как компоненты компилятора>> (выпускная квалификационная работа бакалавра, Шагов М.~А., 2022).
}

Для достижения цели проекта были поставлены следующие задачи:

\begin{itemize}
    \item изучить литературу о структуре и способах разработки существующих компиляторов, а также моделях исполнения программ;
    \item определить подмножество реализуемого синтаксиса языка Python;
    \item разработать архитектуру новых модулей компилятора и организацию взаимодействия между ними;
    \item реализовать описанные модули, произвести их тестирование.
\end{itemize}

Разработка проекта велась коллективом в составе Власова М.~С., Тактаева А.~А., Шагова М.~А., студентов группы 3823М1ПМкн.
Разделы ~\ref{sec:subject_overview}, ~\ref{sec:project_structure}, ~\ref{sec:language} настоящей работы отражают результаты, полученные совместно с равным вкладом всех участников коллектива.
Разделы ~\ref{sec:optree}, ~\ref{sec:converter}, ~\ref{sec:semantizer}, ~\ref{sec:llvmir_codegen} отражают личный вклад автора в проект.

В рамках работы были подробно описаны и реализованы следующие модули:

\begin{itemize}
    \item интерфейс дерева операций,
    \item конвертер для перевода синтаксического дерева в дерево операций,
    \item семантический анализатор дерева операций,
    \item генератор промежуточного представления LLVM IR,
\end{itemize}

а также сопутствующие структуры данных.
